%%% GRAPHS ON POS AND CARD ADOPTION

%%%%%%%%%%%%%%%%%
% PRELIMINARIES %
%%%%%%%%%%%%%%%%%
\documentclass[12pt, aspectratio = 169%, handout%
]{beamer}
\usepackage{amssymb}
\usepackage{amsmath}
\usepackage{bbm}
\usepackage{dsfont}	
\usepackage{epstopdf} %.eps graphs
\usepackage{pdfpages}
\usepackage{curve2e,pict2e}
\usepackage[absolute]{textpos} % to place text/graphics at exact positions
\usepackage{comment}


% ANIMATE GRAPHICS OVER TIME
%\usepackage{graphicx}
\usepackage{graphicx} 
%\usepackage{caption}
%\usepackage{subcaption}
\usepackage{animate}
%\usepackage{caption}
%\usepackage{subcaption}
\usepackage{subfigure}

% FIGURES
\usepackage{tikz}
\usetikzlibrary{math,matrix,calc,positioning}
    

% add space between items http://tex.stackexchange.com/questions/225736/latex-beamer-define-itemsep-globally
\usepackage{xpatch}
\xpatchcmd{\itemize}
  {\def\makelabel}
  {\setlength{\itemsep}{.5ex}\def\makelabel}
  {}
  {}

%font
\usepackage{mathtools}
\usepackage[sfdefault]{roboto} %,light
% \usepackage[activate={true,nocompatibility},final,tracking=true,kerning=true,spacing=true,factor=1100,stretch=10,shrink=10]{microtype}
% \usepackage[scaled]{helvet} %helvetica
\newcommand{\myu}[1]{\underline{\smash{#1}}}
\renewcommand*{\familydefault}{\sfdefault}


	\newcommand{\E}{{\mathbb E}} 
	\newcommand{\R}{{\mathbb R}}
	\newcommand{\Var}{\mathrm{Var}}
	\newcommand{\de}{\delta}
	\newcommand{\be}{\beta}
	\newcommand{\w}{{\omega}}
	\newcommand{\e}{\epsilon}
	\newcommand{\mh}{\hat{\mu}}
	\newcommand{\gh}{\hat{g}}
	\newcommand{\bh}{\hat{\beta}}
	\newcommand{\ga}{\gamma}
	\newcommand{\1}{{\mathbbm{1}}}
	\newcommand{\dif}{\mathrm{d}}
	\newcommand{\half}{{\frac{1}{2}}}
	\newcommand{\im}{\mathrm{i}}
	
	%center itemize
	\newcommand{\tabitem}{%
  		\usebeamertemplate{itemize item}\hspace*{\labelsep}
	}
	
	%create global folder path
	%\newcommand{\folder} {../../Analysis/Stata/output/exhibits1/}
	

%% Change the bg color to adjust your transition slide background color!
\newenvironment{transitionframe}{
  \setbeamercolor{background canvas}{bg=black}
	\setbeamertemplate{footline}[text line]{}
	\setbeamertemplate{frametitle}{%
		\begin{centering}
			\vspace{.425\textheight} % to center vertically
		 {\Large \textcolor{white}{\textbf{\insertframetitle}}}\par
		 \smallskip
		\end{centering}
	}
	\begin{frame}}{%
   \end{frame}\addtocounter{framenumber}{-1}
}

% foreign languages, accents, etc.
\usepackage[T1]{fontenc}
\usepackage[utf8]{inputenc}
\newcommand{\foreign}[1]{\emph{#1}} %change to regular text, emph, etc.

% TABLE FORMATTING
\usepackage{booktabs} %cmidrule
\usepackage{changepage} %to shift table flush left
\usepackage{adjustbox} % to put table in margin and fit to page
% \usepackage{tabu}
\usepackage{xcolor}
\usepackage{siunitx}
\sisetup{%
	%table-format = -1.2,
	detect-all,
	input-symbols = {()[]*},
	group-digits  = false,
	table-align-text-post=false
}

% MATH 
\newcommand{\lfrac}[2]{#1/#2}
\newcommand{\lfracp}[2]{(#1)/(#2)}
\DeclareMathOperator{\asinh}{sinh^{-1}}

% FORMAT BEAMER SLIDES
% Best beamer theme is no theme 
% https://www.quora.com/Presentations-What-are-the-best-beamer-themes
\setbeamertemplate{frametitle}{%
	\begin{centering}
		% \vspace{1pt}
   {\large\textbf{\insertframetitle}}\par
   \smallskip
  \end{centering}
}
\setbeamertemplate{navigation symbols}{}%
	
\setbeamertemplate{footline}%[text line]
	{%
	%\hfill%\strut{%
	\hspace{2.5pt}\raisebox{2.5pt}{\makebox{
        \scriptsize\sf\color{black!60}%
				Lewis (Kelley - IU)
	}}\hfill\raisebox{2.5pt}{\makebox{\scriptsize\sf\color{black!60}\insertframenumber}\hspace{2.5pt} %\hfill%\vspace{2pt}
    %}%
    % \hfill
	}}

\setbeamercolor{title}{fg=black}
\setbeamercolor{frametitle}{fg=black}
\setbeamercolor{normal text}{fg=black}
\setbeamercolor{block title}{fg=black,bg=black!25!white}
\setbeamercolor{block body}{fg=black,bg=black!25!white}
\setbeamercolor{alerted text}{fg=red}
\setbeamercolor{itemize item}{fg=gray}
\setbeamercolor{enumerate item}{fg=gray}
\setbeamercolor{itemize subitem}{fg=gray}
\setbeamercolor{enumerate subitem}{fg=gray}
\setbeamercolor{itemize subsubitem}{fg=gray}
\setbeamercolor{enumerate subsubitem}{fg=gray}
\setbeamercolor{itemize subsubsubitem}{fg=gray}
\setbeamercolor{enumerate subsubsubitem}{fg=gray}
\setbeamercolor{section in toc shaded}{use=structure,fg=structure.fg}
\setbeamercolor{section in toc}{fg=black}
\setbeamercolor{subsection in toc shaded}{fg=black}
\setbeamercolor{subsection in toc}{fg=black}
\setbeamercolor{subsubsection in toc shaded}{fg=black}
\setbeamercolor{subsubsection in toc}{fg=black}
\newcommand{\mycircle}{\raisebox{.2ex}{$\bullet$}}
\newcommand{\mysquare}{\raisebox{.3ex}{\rule{.7ex}{.7ex}}}
\newcommand{\mytriangle}{\raisebox{.2ex}{\scriptsize{$\blacktriangleright$}}}
\newcommand{\myarabic}{\arabic}
\setbeamertemplate{itemize item}{}
\setbeamertemplate{itemize subitem}{\mycircle}
\setbeamertemplate{itemize subsubitem}{\mysquare}
\setbeamertemplate{itemize subsubsubitem}{\mytriangle}
\setbeamertemplate{itemize/enumerate body begin}{\normalsize}
\setbeamertemplate{itemize/enumerate subbody begin}{\normalsize}
\setbeamertemplate{itemize/enumerate subsubbody begin}{\small}
\setbeamertemplate{itemize/enumerate subsubsubbody begin}{\small}

\newenvironment{nscenter} 
 {\parskip=0pt\par\nopagebreak\centering}
 {\par\noindent\ignorespacesafterend}

% BEAMER BUTTONS
\setbeamercolor{button}{bg=white,fg=gray}
\newcommand{\buttonto}[2]{%
	% \vspace*{\fill}
	% \begin{minipage}[t][2\baselineskip][t]{\textwidth}
	\centering
		\raisebox{2\baselineskip}{
		\hyperlink{#1}{\beamergotobutton{#2}}
		}%
	% \end{minipage}
}
\newcommand{\nsbuttonto}[2]{%
	% \vspace*{\fill}
	% \begin{minipage}[t][2\baselineskip][t]{\textwidth}
	\centering
		%\raisebox{2\baselineskip}{
		\hyperlink{#1}{\beamergotobutton{#2}}
		%}%
	% \end{minipage}
}


% Some colors
\definecolor{DarkFern}{HTML}{407428}
\definecolor{DarkCharcoal}{HTML}{4D4944}
\colorlet{Fern}{DarkFern!85!white}
\colorlet{Charcoal}{DarkCharcoal!85!white}
\colorlet{LightCharcoal}{Charcoal!50!white}
\colorlet{AlertColor}{orange!80!black}
\colorlet{DarkRed}{red!70!black}
\colorlet{DarkBlue}{blue!70!black}
\colorlet{DarkGreen}{green!70!black}
\colorlet{AlmostBlack}{black!99.9!violet}

% colors 
\newcommand{\gr}[1]{\textcolor{gray}{#1}}
\newcommand{\T}[1]{\textcolor{blue}{#1}}
\newcommand{\C}[1]{\textcolor{orange}{#1}}
\newcommand{\red}[1]{\textcolor{red}{#1}}
\newcommand{\dc}[1]{\textcolor{DarkCharcoal}{#1}}
\newcommand{\lc}[1]{\textcolor{LightCharcoal}{#1}}
\newcommand{\wt}[1]{\textcolor{white}{#1}}

% For lists
\newcounter{saveenumi}
\newcommand{\seti}{\setcounter{saveenumi}{\value{enumi}}}
\newcommand{\conti}{\setcounter{enumi}{\value{saveenumi}}}

%additional formatting
\setbeamersize{text margin left=16pt,text margin right=16pt} 

% SPANISH
% \usepackage[spanish]{babel}
\newcommand{\myforeign}[1]{{#1}} % change to {{#1}} for no emphasis; {\textbf {#1}} for bold, etc.


\setbeamertemplate{caption}{\raggedright\insertcaption\par}


%-------------------------------------------------------------------------------%
% Estout related things
%-------------------------------------------------------------------------------%
% http://www.jwe.cc/2012/03/stata-latex-tables-estout/
%------------------------------------------------------------------------------%
\newcommand{\sym}[1]{\rlap{#1}}% Thanks to David Carlisle

\let\estinput=\input% define a new input command so that we can still flatten the document

\newcommand{\estwide}[3]{
\vspace{.75ex}{
\begin{tabular*}
{\textwidth}{@{\hskip\tabcolsep\extracolsep\fill}l*{#2}{#3}}
\toprule
\estinput{#1}
\bottomrule
\addlinespace[.75ex]
\end{tabular*}
}
}

\newcommand{\estauto}[3]{
\vspace{.75ex}{
\begin{tabular}{l*{#2}{#3}}
\toprule
\estinput{#1}
\bottomrule
\addlinespace[.75ex]
\end{tabular}
}
}



%%%%%% title page %%%%%%%
\title{\Large\textbf{%
	Coordinated Betting by Multi-Fund Managers \\
	\large (Gelly Fu)
%
}}
\author{\Large
	Brittany Lewis \\
	Kelley School of Business - IU\\
	\vspace{2em}
	\small Discussion for MFA 2020 \\
	\small $69^{th}$ Annual Meeting
}
%\institute{Northwestern}


%\date{\large%
%October 14, 2019}

\begin{document}


%%%%%%%%%%%%% TITLE PAGE %%%%%%%%%%%%%%%
{
\setbeamercolor{background canvas}{bg=black}
\setbeamercolor{title}{fg=white}
\setbeamercolor{author}{fg=white}
\setbeamercolor{date}{fg=white}
\setbeamertemplate{footline}{} 
\begin{frame}
	\titlepage
\end{frame}
}
\addtocounter{framenumber}{-1}
%%%%%%%%%%% end of title page %%%%%%%%%%

\begin{frame}{Motivation}
\begin{itemize}
	\item
	\begin{itemize} \itemsep3.5ex
		\item Mutual funds larger share of the financial markets than ever before (Falato, Goldstein, 			Hortacsu 2020)
		\item Mutual funds have incentive to outperform to attract funds (Massa, Patgiri 2008)
		\item Mutual fund managers' compensation contracts reward outperformance but do not penalize underperformance (Ma, Tang, Gomez 2019)
		%\item Drivers of outflows of treasuries during COVID (Ma, Xiao, Zeng 2020)
		\item Do fund managers seek outperformance in a way that harms investors?
	\end{itemize}
\end{itemize}
\end{frame}

\begin{frame}{Summary of Main Results}
\begin{itemize} \itemsep2ex
		\item Paper proposes new agency problem where managers manage 2 or more funds and 				maximize the probability that one fund outperforms 
			\begin{itemize}
				\item To do this:
					\begin{enumerate}
						\item Presents model where multi-fund manager maximizes her own 								consumption by selecting negatively correlated stocks across her funds 
						\item Combines 3 data sources to study whether theory is borne out in the 							data
					\end{enumerate}
			\end{itemize}
		\item \textbf{Main result:} returns less correlated between two funds managed by same manager relative 			to other matched funds
		\item Other results: These managers take more risk, trade more often, are more heavily weighted 			in volatile sectors such as finance and manufacturing rather than less volatile sectors such 			as telecom and energy
\end{itemize}
\end{frame}

\begin{frame}{Robustness Tests and Consistent Outcomes}
\begin{itemize}
	\item \textbf{Robustness Tests}
	\begin{itemize}
		\item \small SameStyle
		\item \small TeamSample
		\item \small Placebo - Managers similar but not identical
			\begin{itemize}
				\item At least one manager the same and at least one unique to the 2 funds
			\end{itemize}
	\end{itemize}
	\item \textbf{Consistent Outcomes - funds that engage in negative correlation}
		\begin{itemize}
		\item \small Strategically coordinate investments in different industries - opposite portfolio weights 				in more volatile industries (manufacturing and finance rather than telecom and energy)
		\item \small Take more positions skewed toward small cap in one fund and toward 							large cap in another fund
		\item \small Risk Taking
			\begin{itemize}
				\item Have 0.31\% higher 	volatility			
				\item Invest in more lottery like stocks 
			\end{itemize}
	\end{itemize}
\end{itemize}
\end{frame}

\begin{frame}{Main Comment: Result Depends on Matching}
\begin{itemize}
	\item \textbf{ Matching methodology }
	\begin{itemize} \itemsep1ex
		\item  Identify funds managed by same manager - funds $i$ and $j$
		\item  Identify common stocks in both as $C_{i,j}$
		\item  Identify unique stocks in j relative to i as $U_{i,j}$
		\item  Match $j$ to the universe of funds in same investment style and size quintile 
			\begin{itemize} 
				\item Call each matched fund $M$
			\end{itemize}
		\item  Generate synthetic portfolio $M^*$ using fund $j$ and $M$ holdings
			\begin{itemize} 
				\item $M^*$ splices together $C_{i,j}$ and $U_{i,M}$
			\end{itemize}
		\item  Measure $corr(i,j)$ relative to $corr(i,M^*)$
	\end{itemize}
\end{itemize}
\end{frame}		

\begin{frame}{Main Comment: Result Depends on Matching Cont'd}
\vspace*{-1em}
	\begin{itemize}	
		\item
		\begin{itemize} 
			\item \small$M^*$ is synthetic portfolio composed of $C_{i,j}$ and $U_{i,M}$
			\item \small Result: $corr(i,j) < corr(i,M^*)$
				\begin{itemize}
					\item $\Rightarrow$ \textbf{more discussion on the matching process and synthetic fund}
				\end{itemize}
		\end{itemize}
	\end{itemize}
	
	\begin{itemize}
			\item \small Result depends on matching process
				\begin{itemize}
%					\item \small If $M$ is matched to entire fund $j$ and $j$ and $i$ have the same 								manager, then $C_{i,j} \Rightarrow M$ likely to be similar to $i$
%					\begin{itemize}
%						\item Because of reversion to the mean, $U_{M,i}$ likely to be more 												correlated with $i$ than the stock bets in $j$
%						\item Good robustness check could be to match $M$ to $U_{j,i}$
%					\end{itemize}
					\item How similar are $C_{M,i}$ and $C_{j,i}$? 
					\item Do they contain the same number of stocks, do they have the same average return?
					\begin{enumerate}
						\item Yes: could validate swapping $C_{M,i}$ with $C_{j,i}$
						\item No: it may be that the interplay between $C_{M,i}$ and $U_{M,i}$ is 											important to track $j$ \\
							$\Rightarrow$ swapping $C_{M,i}$ with $C_{j,i}$ could overstate $M^*$'s correlation 							with $i$ and drive result that $corr(j,i) < corr(M^*,i)$
					\end{enumerate}
				\end{itemize}
		\end{itemize}

\end{frame}


\begin{frame}{Simple Example}
\vspace*{-1em}
	\begin{itemize}	
		\item
		\begin{itemize} 
			\item Fund $i$ has 3 stocks: Target, Walmart, Nike
			\item Fund $j$ has 3 stocks: Target, Walmart, United
			\item Fund $M$ has 3 stocks: Walmart, United, Delta
				\begin{itemize}
					\item Fund $M$ selected to be similar to $j$, overlap with $j$ is Walmart, United
					\item $M$'s overlap with $i$ however is only Walmart
					\item Methodology creates $M^*$ by swapping $M$'s Walmart for $j$'s Target and 							Walmart
				\end{itemize}
		\end{itemize}
	\end{itemize}
	
	\begin{itemize}
		\item
		\begin{itemize}
			\item Now depending on how the $U_{M^*,i}$ weighting works, it could matter how United 				and Delta are weighted in the synthetic portfolio
			\item Extreme example where $U_{j,i}$'s overlap with $U_{M,i}$ is weighted at 0
				\begin{itemize}	
					\item Only consider the new bet: Delta
				\end{itemize}
		\end{itemize}
	\end{itemize}

\end{frame}

\begin{frame}{Imagine the Following Stocks}
\begin{center}
\begin{tabular}{|c|c|c|}
\hline
Fund $i$ Holdings & Fund $j$ Holdings & Fund $M$ Holdings \tabularnewline
\hline
\hline 
$\textcolor{red}{Target}$ & $\textcolor{red}{Target}$ & \textcolor{AlertColor}{Delta} \tabularnewline
\hline  
$\textcolor{red}{Walmart}$ & $\textcolor{red}{Walmart}$ & \textcolor{red}{Walmart} \tabularnewline
\hline 
$Nike$ & $\textcolor{blue}{United}$  & \textcolor{blue}{United} \tabularnewline
\hline 

\end{tabular}
\par\end{center}
\end{frame}

\begin{frame}{Simple Example}
\vspace*{1em}
\small
\begin{columns}[t]
	\begin{column}{0.5\textwidth}
	%\column{.5\textwidth}
	\centering
	\vspace*{-2.5em}
		\begin{center}
                        \begin{tabular}{|c|c|c|}
                        \hline
                         Fund i Return & Fund j Return \tabularnewline
                        \hline
                        \hline 
                        \textcolor{red}{(Target) 2} & \textcolor{red}{(Target) 2}  \tabularnewline
                        \hline  
                        \textcolor{red}{(Walmart) 4} & \textcolor{red}{(Walmart) 4}  \tabularnewline
                        \hline 
                        (Nike) 3 & \textcolor{blue}{(United) -4}  \tabularnewline
                        \hline 
                        \end{tabular}
                        \par\end{center}
                        correlation $= 0.24$
	\end{column}
	\column{.5\textwidth}
	\onslide<2->{ % 2-, the minus means stay forever after, could say only stay from 2 - 3 fo example
	\centering
	\vspace*{-2.5em}
	\begin{center}
                    \begin{tabular}{|c|c|c|}
                    \hline
                     Fund i Return & Fund M Return \tabularnewline
                    \hline
                    \hline 
                    \textcolor{red}{(Target) 2} & \textcolor{AlertColor}{(Delta) -2}  \tabularnewline
                    \hline  
                    \textcolor{red}{(Walmart) 4} &  \textcolor{red}{(Walmart) 4} \tabularnewline
                    \hline 
                    (Nike) 3 &  \textcolor{blue}{(United) -4}   \tabularnewline
                    \hline 
                    \end{tabular}
                    \par\end{center}
                    correlation $= -0.72$
		}
	\end{columns}

        \onslide<3->{
        \begin{center}
        \begin{tabular}{|c|c|c|}
        \hline
         Fund i Return & Fund $M^*$ Return \tabularnewline
        \hline
        \hline 
        \textcolor{red}{(Target) 2} & \textcolor{red}{(Target) 2}  \tabularnewline
        \hline  
        \textcolor{red}{(Walmart) 4} & \textcolor{red}{(Walmart) 4}  \tabularnewline
        \hline 
        (Nike) 3 & \textcolor{AlertColor}{(Delta) -2}  \tabularnewline
        \hline 
        & \textcolor{blue}{(United) -4}  \tabularnewline
        \hline 
        \end{tabular} \\
        correlation $= 0.33$ \\
        \par\end{center}
        }
        \onslide<4->{ 
        \vspace{0.25em}
        \begin{eqnarray*}
        corr(j,i) = 0.24 &<& corr(M^*,i) = 0.33 }\\
        \onslide<5->{ \textbf{But} \ corr(j,i) = 0.24 &>& corr(M,i) = -0.72 \rightarrow \textbf{Reverses the result} }
        \end{eqnarray*}

\vspace*{2em} % +/- # of text lines of vertical space (em is a text line)
\end{frame}

%\begin{frame}{Macro ImpliTargetions}
%\begin{itemize}
%	\item 
%	\begin{itemize} \itemsep3.5ex
%		%\item think about how comparing style difference from i to j versus from i to M and return 				correlation from i to j and i to $M*$ correspond to each other. (is that what he is doing or is 			he using $M*$ for both? and what do I think about this?)
%	\end{itemize}
%\end{itemize}
%\end{frame}


\begin{frame}{Other Comments - Performance Analysis}
\begin{itemize}
	\item Managers engaging in this strategy are 39\% more likely to produce star funds
		\begin{itemize} 
			\item Managing two or more is already signal that you are a better manager
			\begin{itemize}
				\item Coordination is at the manager level. Worried about bias at the manager level - 					MBA, PhD, past performance at the manager (rather than fund) level 
			\end{itemize}
		\end{itemize}
	%\item spillovers across Multi-Funds, not across Fund Families (which is in contrast to Nanda 2004)
\end{itemize}
\end{frame}

\begin{frame}{Conclusion}
\begin{itemize}
	\item
		\begin{itemize} \itemsep3.5ex
		\item Important question
		\item Interesting approach
		\item Additional analysis explaining matching process and creation of synthetic portfolio $M^*$ 				would be valuable
		\item Additional controls for manager characteristics would be valuable
		\end{itemize}
\end{itemize}
\end{frame}

\end{document}
